\documentclass[11pt, a4paper]{article}
\usepackage[utf8]{inputenc}
\usepackage[english,russian]{babel}
\usepackage[colorlinks=true,linkcolor=blue]{hyperref}
\usepackage{listings}
\usepackage{color}
\usepackage{verbatim}

%\usepackage{tocloft} % it requires package texlive-latex-extra in Debian

\usepackage{indentfirst}

%
% for tocloft
%
%\renewcommand{\cftsecaftersnum}{.}
%\renewcommand{\cftsubsecaftersnum}{.}

\renewcommand{\thesection}{\arabic{section}.}



\definecolor{mygreen}{rgb}{0,0.6,0}
\definecolor{mygray}{rgb}{0.5,0.5,0.5}
\definecolor{mymauve}{rgb}{0.58,0,0.82}

\lstset{ %
  backgroundcolor=\color{white},   % choose the background color; you must add \usepackage{color} or \usepackage{xcolor}
  basicstyle=\footnotesize,        % the size of the fonts that are used for the code
  breakatwhitespace=false,         % sets if automatic breaks should only happen at whitespace
  breaklines=true,                 % sets automatic line breaking
  captionpos=b,                    % sets the caption-position to bottom
  commentstyle=\color{mygreen},    % comment style
  deletekeywords={...},            % if you want to delete keywords from the given language
  escapeinside={\%*}{*)},          % if you want to add LaTeX within your code
  extendedchars=true,              % lets you use non-ASCII characters; for 8-bits encodings only, does not work with UTF-8
  frame=single,                    % adds a frame around the code
  keepspaces=true,                 % keeps spaces in text, useful for keeping indentation of code (possibly needs columns=flexible)
  keywordstyle=\color{blue},       % keyword style
  language=C,                      % the language of the code
  morekeywords={*,...},            % if you want to add more keywords to the set
  numbers=left,                    % where to put the line-numbers; possible values are (none, left, right)
  numbersep=5pt,                   % how far the line-numbers are from the code
  numberstyle=\tiny\color{mygray}, % the style that is used for the line-numbers
  rulecolor=\color{black},         % if not set, the frame-color may be changed on line-breaks within not-black text (e.g. comments (green here))
  showspaces=false,                % show spaces everywhere adding particular underscores; it overrides 'showstringspaces'
  showstringspaces=false,          % underline spaces within strings only
  showtabs=false,                  % show tabs within strings adding particular underscores
  stepnumber=2,                    % the step between two line-numbers. If it's 1, each line will be numbered
  stringstyle=\color{mymauve},     % string literal style
  tabsize=2,                       % sets default tabsize to 2 spaces
  title=\lstname                   % show the filename of files included with \lstinputlisting; also try caption instead of title
}


\title{Задание практикума: командная оболочка операционной системы}
\author{Сальников А.Н.}
\date{}

\begin{document}

\maketitle

\tableofcontents

\section{Общее описание}

Требуется написать программу на языке \textbf{\textit{Си}}, осуществляющую частичную 
эмуляцию командной оболочки (shell). Примеры программных оболочек:
{\selectlanguage{english}
Windows Console (cmd)\cite{cmd}, Windows Power Shell\cite{Power_Shell}, bash\cite{guide-eng},
tcsh\cite{tcsh}, zsh\cite{zsh}.}

Командная оболочка представляет собой интерактивную программу. Поток команд берется со 
стандартного потока ввода. Команды shell-у задаются пользователем через командую строку. 
Считывать команды нужно в цикле до тех пор, пока на очередной итерации не будет получен конец 
файла (детектируется при помощи константы EOF или проверкой соответствующих возвращаемых значений 
функции чтения из файлов/потоков) либо не будет введена команда \textbf{exit}. Перед считыванием 
нужно вывести на экран приглашение к набору команд. Например приглашение может быть таким: 
\textit{``vasia''} далее символ \textit{\$} и следующий за ним пробел).
Длина считываемой строки ограничена только размером
виртуальной памяти доступной процессу.

Пример приглашения ввода команд (здесь пользователь уже набрал некторорую команду):
{\small
\begin{verbatim}
vasia$ echo "We found: "; find /home -name \*${USER}\* 2>/dev/null| \
       wc -l; echo " files"
\end{verbatim}
}

Завершение shell происходит либо по закрытию стандартного потока ввода, либо в случае 
выполнения встроенной команды  \textbf{exit} (Реализованы должны быть оба варианта).

Программа должна корректно обрабатывать все ошибки, выдавая осмысленные информационные сообщения о них
в стандартный поток ошибок. В том числе обрабатывать ошибки неправильного синтаксиса командной строки. 

\section{Описание языка команд shell}

Командная оболочка shell выполняет группу работ, где каждая работа отделяется от другой 
символами перевода строки (`{\textbackslash}n` в языке \textbf{\textit{Си}}) и символом `;'.
Внутри работы могут встречаться команды, их аргументы и перенаправления ввода/вывода, 
которые могут разделятся символами конвейера '|'. 

В строке могут встречаться следующие конструкции:
\begin{itemize}
\item текст в кавычках (позволяют вставить пробел в аргумент программы). Кавычки бывают двух видов: 
      двойные и одинарные. Отличие текста в двойных ковычках от текста в одинарных заключается в 
      том, что в двойных кавычках производится подстановка переменных, а в одинарных нет.

       Например:\newline
       \verb#echo "I am ${USER}"# напечатает для пользователя c именем <<vasia>> \newline
         \verb#I am vasia# \newline
       \verb#echo 'I am ${USER}'# напечатает \newline \verb#I am ${USER}#

    \item  \textit{экранирование символа}. Осуществляется при помощи символа `\textbackslash'. 
       Символ после обратного слэш не будет иметь <<служебного>> смысла. Например обратный слэш 
       позволяет поставить кавычку внутри кавычек. Последовательность `\textbackslash\textbackslash'
       позволяет ввести обычный одинарный слэш. Символ `\textbackslash' может так же наоборот 
       означать наличие специального смысла символа в ситуации, когда без этой конструкции особого
       смысла небыло. Если символ `\textbackslash' стоит в конце строки (следующий символ в файле 
       перевод строки), то это означает, что перевод строки <<съедается>>, а строка на новой 
       строчке на самом деле является продолжением текущей.

   \item  \textit{Коментарии}. Если во входной строке встречается символ \textit{`\#'}, который 
       находится не внутри кавычек одинарных или двойных и не экранированн обратным слэшом,
       то все символы, стоящие после решётки до конца строки игнорируются.

   \item  \textit{Подстановка значений переменных}. Перед тем как интерпретировать команду необходимо 
       подставить в командную строку значения переменных. Могут быть служебные переменные, а могут быть 
       пользовательские переменные. Данное задание предполагает только подстановку служебных переменных.
       Значение переменной --- это некотороый текст, который сопоставлен имени пременной.  
       В случае встречи в строке конструкции вида:
       {\small 
       \begin{verbatim}
          ${ИМЯ_ПЕРЕМЕННОЙ}
       \end{verbatim}
       }
       вместо этой конструкции, в строку, будет подставлена строка, являющаяся значением переменной.
\end{itemize}

\section{Команды и запуск команд}

Команды можно разделить на 2 группы. Первая -- встроенные команды shell. Часто они не требуют запуска
отдельных процессов (их реализация является частью кода shell\footnote{Однако они могут присутствовать 
в конвейере, тогда для них необходимо создавать отдельные процессы, перенаправлять ввод/вывод и т.п., 
но вместо вызова exec, нужно вызвать функцию в коде shell, реализующую встроенную команду.}),
например команда \textbf{cd} или \textbf{pwd}. Внешние команды shell --- это обычные
программы, которые запускаются так, что каждая программа оказывается в своём отдельном процессе.
На всторенные команды не может быть вызван exec.

Команде можно указать список её аргументов. Например:
{\small 
\begin{verbatim}
          ls -l -a ../.. #конец аргументов
\end{verbatim}
}
передаст программе ls после её запуска:
\newline в argv[0] ``ls'',
\newline в argv[1] ``-l'',
\newline в argv[2] ``-a'',
\newline в argv[3] ``../..''.

После списка аргументов прогаммы допускеается указывать символы \verb#<# и \verb#>#, которые позволяют 
считать стандартный ввод из файла или вывести стандартный вывод в файл, а также последовательность
символов \verb#>>#, что позволяет дописать вывод программы в конец указанного файла. 
Например:\newline
{\small
\begin{verbatim}
        # Перенаправим ввод команде cat 
        # из файла file.in и допишем 
        # файл file.out
        cat < file.in >> file.out
\end{verbatim}
}

После перенаправлений ввода/вывода допускается указать символ \verb#&#, который означает запуск 
команды в фоновом режиме. Подробнее про фоновый режим далее.

При помощи символа \verb#|# организуется конвейер. Смысл символа таков, что команда слева от символа
делает свой стандартный поток вывода стандартным потоком ввода для команды справа от символа. 
И так продолжается дальше по цепочке, пока <<вертикальные палки>> не закончатся. В случае, если имело 
место перенаправление ввода/вывода, то приоритет отдаётся именно перенаправлению, а не конвейеру. В
результате роцесс подключённый к конвейеру справа получит себе конец файла, либо процесс слева 
будет некому <<слушать>> из конвейера и он вероятно получит себе \textit{SIGPIPE}.

Символ \verb#&# допускается указывать только в конце определения исполняемой работы shell. То есть 
до символа задающего конвейер \verb#&# указать нельзя.

\section{Режим переднего плана и фоновый режим}

В режиме переднего плана (foreground) работы исполняются последовательно одна за другой. 
Каждая работа на время своего выполнения получает терминал в свой монопольный доступ. 
После исполнения одной или нескольких работ (в случае, если они разделены точкой с запятой)
пользователю выдаётся стандартное приглашение ко вводу следующего набора команд. 
При этом, нажатие \textit{Ctrl+c} должно приводить к посылке сигнала текущей выполняющейся работе,
но не самому процессу реализующему shell. По нажатию \textit{Ctrl+z} работа должна 
приостанавливаться, после чего выдаётся приглашение shell, которое позволяет запустить новую порцию
команд или продолжить выполнение приостановленной ранее работы путём указания её номера во 
встроенной команде \textbf{fg} или \textbf{bg}. Список имеющихся работ можно посмотреть при 
помощи команды \textbf{jobs}.

Работа запущенная в фоновом режиме (background) не должна обращаться к терминалу. В случае обращения 
к терминалу процесс из работы запущенной в фоновом режиме должен быть приостановлен (не завершён). 
Если работа запускается в фоновом режиме, то процессы не должны получать \textit{SIGINT} 
в случае нажатия \textit{Ctrl+c} на клавиатуре. В случае запуска работы в фоновом режиме shell 
не вместо ожидания завершения такойц работы, сразу либо выдаёт приглашение ко вводу следующего 
набора команд, либо выполняет следующую работу (например они были разделены точкой с запятой). 
В случае завершения фоновой работы shell информирует об этом пользователя, выдавая соответствующее
сообщение в стандартный поток ошибок основным процессом shell
(тот, который был изначально при запуске приложения).

\section{Запуск команд из истории}

Шеллы позволяют использовать короткое мнемоническое имя для запуска команды из истории запусков.
Для этого используется конструкция вида \verb#!100500#, здесь 100500 - это номер команды в истории команд. 
Конструкция должна сработать как подстановка переменной, тоесть в то место, где встретилась конструкция подствляется 
вводившаяся когда-то команда. Дальше могут идти конструкции с точкой-запятой, перенаправление ввода-вывода
и т.п. Пример:
\begin{verbatim}
vasia$ history | grep gcc 
 1022  gcc -E  test_abc.c
 1023  gcc -E  test_abc.c > /tmp/filo.c
 1025  gcc -S  test_abc.c
 1027  gcc -c  test_abc.c
 1029  gcc -o test_abc test_abc.c
 1032  gcc -o test_abc test_abc.c -lm
 1035  gcc -o test_abc test_abc.c -lm
 1206  gcc -g redactor.c
 1229  gcc -g ilya.c 
 1476  gcc prog32.c 
 1852  gcc terminal_manipulation.c 
 1855  gcc term_mode_change.c 
 1859  gcc term_mode_change.c 
 1862  gcc term_mode_change.c 
 1964  gcc -Wall -ansi -pedantic -g main9.c 
 1966  gcc -Wall -ansi -pedantic -g main9.c 
 1968  gcc -Wall -ansi -pedantic -g main9.c 
 1994  if gcc foo ; then echo "OK"; fi
 1995  if gcc foo ; then echo "OK"; else echo "fooo";  fi
 2005  history | grep gcc

vasia$ !1862 ; !1966 
gcc term_mode_change.c  ; gcc -Wall -ansi -pedantic -g main9.c  
\end{verbatim}

\section{Дополнительные требования}

Требуется реализовать встроенные команды:
\begin{itemize}
\item  \textbf{cd} -- смена текущего каталога
\item  \textbf{pwd} -- выдача текущего каталога в стандартный поток вывода
\item  \textbf{jobs}, \textbf{fg}, \textbf{bg} -- выдача списка активных в текущий момент работ
\item  \textbf{exit} -- выход из shell
\item  \textbf{history} -- команда показывающая историю команд
\item export -- передача переменных окружения в запускаемые процессы из текщего shell
\end{itemize}

Требуется реализовать как встроенные команды следующие внешние программы:
\begin{itemize}
\item
    \textbf{cat} --  с именем \textbf{mcat}, где в качестве параметра либо ничего не указывается,
    либо указывается имя файла (полный набор параметров реализовывать не нужно).
\item 
    \textbf{sed} -- c именем \textbf{msed}. Реализовать минимальный вариант 
    подстановки по образцу, где в качестве шаблона используется просто текст 
    без спецсимволов, которые позволяют задавать синтаксис регулярных выражений 
    \cite{regex}. Первый аргумент задаёт строку образец, второй ааргумент -- 
    строка на которую будет заменён образец. Необходимо осуществлять замену 
    всех вхождений образца в строке. Символ `\textasciicircum' означает, что второй аргумент 
    программы необходимо приписать вначало всех строк. Cимвол `\$' означает, что второй аргумент 
    программы необходимо приписать в конец всех строк. Внутри строк, на которые заменяем 
    может встречаться перевод строки, который задаётся '\\n'.
\item
    \textbf{grep} -- с именем \textbf{mgrep} Реализовать минимальный вариант. Формат шаблона такой 
        же как для предыдущей команды \textbf{msed}. Дополнительно возможны конструкции \textit{.*}
        -- означает произвольное количесттво любых символов, в том числе пустое; и \textit{.+} --
        означает произвольное количество любых символов, но не пустое.
\end{itemize}

Требуется реализовать подстановку любых переменных, которые перед запуском выставлены в переменных 
окружения, в том числе служебных:
\begin{itemize}
\item \verb#$цифра# -- подстановка соответствующего аргумента командной строки самого shell.
\item \verb@$#@ -- число параметров переданное shell
\item \verb@$?@ -- значение статуса последнего завершившегося процесса в последней выполнившейся
                   работе переднего плана.
\item \verb#${USER}# -- login пользователя.
\item \verb#${HOME}# -- домашний каталог пользователя.
\item \verb#${SHELL}# -- имя shell. Путь до того места в файловой системе, где находится 
                         исполняемый файл с ним. (В качестве плохо работающего <<костыля>> 
                         допускается реализация в виде подстановки argv[0]).
\item \verb#${UID}# -- идентификатор пользователя
\item \verb#${PWD}# -- текущий каталог
\item \verb#${PID}# -- pid shell
\item \verb#$HOSTNAME# -- имя машины на которой запускается shell   
\end{itemize}
%О работе с переменными окружения можно прочитать здесь:~\cite{getenv}.

\newpage
\section{Рекомендации по написанию кода задания}

На начальном этапе при выполнении задания необходимо научиться считывать командные строки 
шелл и сохранять их в оперативной памяти (не выполняя команды). 
Рекомендуется данные в памяти хранить как массив структур следующего вида:

\begin{lstlisting}

struct program
{
    char* name;
    int number_of_arguments;
    char** arguments;
    char *input_file, *output_file; /* NULL - not redirected */
    int output_type; /* 1 - rewrite, 2 - append */
};

struct job
{
    int state; /* stopped, background, foreground */
    struct program* programs;
    int number_of_programs;
    pid_t process_group;
    pid_t *pids;
};

\end{lstlisting}

После того, как убедились, что разбор команд происходит правильно можно приступать к 
реазлизации запуска действий связанных с командами.
Для реализации команды \textbf{history}
целесообразно использовать очередь с ограничением на её длину. 

Для корректного отслеживания завершающихся процессов полезно определить действие на приход 
сигнала \textit{SIGCHLD}.

Не стоит надеятся, что служебные переменные, такие как \verb#${HOME}#  будут для вас выставлены.
Задача Shell как раз заключается в том, что Shell эти переменные определяет сам и выставляет для 
запускаемых из него программ.


\begin{thebibliography}{50}
\bibitem{cmd}
    Официальная документация на команду cmd на сайте Microsft:
    \url{htps://www.microsoft.com/resources/documentation/windows/xp/all/proddocs/en-us/ntcmds.mspx?mfr=true}.

\bibitem{Power_Shell}
    Сайт проекта Microsoft Power Shell:  \url{https://msdn.microsoft.com/en-us/powershell}.

\bibitem{tcsh}
    Сайт tcsh: \url{http://www.tcsh.org/Welcome}.

\bibitem{zsh}
    Сайт одного из наиболее полного всем чем только можно шелла:
    \url{http://zsh.sourceforge.net/Doc/}.

\bibitem{man}
    Справочная страница с описанием реализации shell /bin/bash: ``man bash''.

\bibitem{guide-rus}
    Перевод руководства по bash скриптам: \url{http://www.opennet.ru/docs/RUS/bash\_scripting\_guide}.

\bibitem{guide-eng}
    Более полное руководство по bash \url{http://www.gnu.org/software/bash/manual/bashref.html}
    
\bibitem{regex}
    Справочная страница по регулярным выражениям: ``man 7 regex''

\bibitem{getenv}
    Справочная страница по функции getenv ``man getenv''

\end{thebibliography}

\end{document}

